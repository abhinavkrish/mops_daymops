\section{Risk Reduction}

Though the core algorithms of MOPS have been implemented in
LSST-appropriate style, further research and development are needed.



\subsection{Long Duration Survey Performance}

Current simulations cover fairly short time periods, and therefore
emphasize the problem of initial object discovery.  In the course of
the full survey, we expect that many detected sources will be
attributed to already-discovered objects.  Because initial object
discovery phases are relatively expensive and ephemeris calculation is
relatively fast, we expect that the resource usage of the system will
decline over time, as more objects are discovered and the size of
input catalogs is reduced.  This expectation needs to be verified and
quantified.

Attribution, precovery and Moving Object management and refinement of the
Moving Object table are not yet implemented in LSST-compliant software.
Developing this software should be a significant development task.
However, we hope that by using the algorithms from the PanSTARRS MOPS
we can avoid any significant research tasks.

To test this software, we will need to generate simulated input
catalogs which span longer time periods.  Accomplishing this will
require either significant compute-resources or improved tools for
generating input catalogs.


\subsection{Future Software Development Tasks}

\subsubsection{Filtering on Trailing for Near-Earth-Object Searching}

\label{neosTrailing}

Near-Earth Objects tend to have the highest sky-plane velocity.  This
presents a significant challenge; as we increase the maximum velocity
limit of our tracklet generation, the potential for mislinkage
increases significantly, leading to higher numbers of tracklets and
increased costs.  

Fortunately, fast-moving NEOs will generate visible trails in our
images.  By requiring all tracklets to show trails consistent with
their apparent sky-plane velocity, we expect that it will be possible
to filter most false tracklet linkages, thus rendering the problem of
NEO searching manageable.

The ability to filter on trailing is dependent almost entirely on our
ability to correctly identify trails in images.  Currently, the
ability of image processing to detect trails is not well quantified.  To
remedy this, we will need to generate simulated images which include
asteroid trails and send them to image processing; further refinement
of image processing algorithms may be neccesary.


\subsubsection{Dense and Clustered Noise}

\textbf{TBD}


