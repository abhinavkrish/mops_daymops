\section{Risk Reduction}

Though the core algorithms of MOPS have been implemented in
LSST-appropriate style, further research and development are needed.



\subsection{Long Duration Survey Performance}

Current simulations cover fairly short time periods, and therefore
emphasize the problem of initial object discovery.  In the course of
the full survey, we expect that many detected sources will be
attributed to already-discovered objects.  Because initial object
discovery phases are relatively expensive and ephemeris calculation is
relatively fast, we expect that the resource usage of the system will
decline over time, as more objects are discovered and the size of
input catalogs is reduced.  This expectation needs to be verified and
quantified.

Attribution, precovery and Moving Object management and refinement of the
Moving Object table are not yet implemented in LSST-compliant software.
Developing this software should be a significant development task.
However, we hope that by using the algorithms from the PanSTARRS MOPS
we can avoid any significant research tasks.

To test this software, we will need to generate simulated input
catalogs which span longer time periods.  Accomplishing this will
require either significant compute-resources or improved tools for
generating input catalogs.


\subsection{Future Software Development Tasks}

\subsubsection{Filtering on Trailing for Near-Earth-Object Searching}

\label{neosTrailing}

Near-Earth Objects tend to have the highest sky-plane velocity.  This
presents a significant challenge; as we increase the maximum velocity
limit of our tracklet generation, the potential for mislinkage
increases significantly, leading to higher numbers of tracklets and
increased costs.  

Fortunately, fast-moving NEOs will generate visible trails in our
images.  By requiring all tracklets to show trails consistent with
their apparent sky-plane velocity, we expect that it will be possible
to filter most false tracklet linkages, thus rendering the problem of
NEO searching manageable.

The ability to filter on trailing is dependent almost entirely on our
ability to correctly identify trails in images.  Currently, the
ability of image processing to detect trails is not well quantified.  To
remedy this, we will need to generate simulated images which include
asteroid trails and send them to image processing; further refinement
of image processing algorithms may be neccesary.


\subsubsubsection{Distributed LinkTracklets} 

In case the implicit memory sharing is not sufficient (say, because
distributed shared memory systems do not provide adequate
performance), it may be necessary to write an explicitly distributed
linkTracklets.  This was attempted by a CS Master's student, Matt
Cleveland, working with Prof. Dave Lowenthal.  The design was
well-thought out but complex, leading to slow implementation.  The
distributed version was forked off before a variety of changes to the
serial version were made, and never merged with those changes.

The distributed linkTracklets, the master node held only the tracklets
of the endpoint trees, and the higher levels of the endpoint and
support trees.  The master node would attempt the linking algorithm on
the higher levels of the tree until it reached a terminal point; it
would then attempt to predict the amount of work needed to complete
the linking and save the state of the searching and estimated cost to
a buffer.  Periodically, the work items in the buffer are distributed
to worker nodes, with attention given to load distribution as well as
cache issues, attempting to minimize the amount of data which must be
transferred to worker nodes.


\subsubsection{Trivial Code Changes}
\paragraph{Reducing Memory Bloat in KD-Tree Libraries}

Because of the size of the data sets in MOPS, memory consumption is a
frequent problem.  Though this cannot necessarily be addressed
exclusively through improvements in software implementation alone, the
overhead could be reduced significantly.

Lack of attention to the memory size in the initial implementation of
the KD-Tree library have resulted in needless memory bloat of the
trees.  Each node uses a series of \texttt{std::vector}s which could
be easily replaced with pointers or C-style arrays.  On a 64-bit
architecture, \texttt{std::vector} has an overhead of at least 24
bytes, where pointers and fixed-size arrays have an overhead of just 8
bytes.  Currently, KD-Trees use four vectors, meaning an overhead of
96 bytes per node.

Further modification could further reduce the size of KD-Tree nodes;
for instance, rather than storing $k$ in an integer, it could made a
template argument so the constant is ``baked in'' at compile time and
would not need to be stored.  

\paragraph{Removing Critical Sections in Parallel Implementations}
In all the multithreaded software, we maintain a single output vector
(or set) to which all threads write.  To prevent conflicts, these
writes must be wrapped in critical sections.  Critical sections impose
synchronization costs, which may become significant when threads
execute on separate boards and must communicate over the network (as
in vSMP or other OS-level distributed shared memory systems).

To reduce overhead, each thread should really maintain its own output
vector, which could be used without any critical section and merged
later, after all worker threads have completed.  This can be
accomplished in the master thread before the parallel section, by by
using \texttt{omp\_get\_num\_threads} to determine the number of
threads, and then allocating one result vector per thread.  Then, the
individual worker threads should write to their allocated results
vector and these can be accumulated later.  For reasons of time, this
has not yet been implemented, but could be done quickly.

\subsubsection{Significant Changes in CollapseTracklets}
\paragraph{Motion Vectors} 
Currently, collapseTracklets parameterizes motion using location,
angle, and velocity.  However, for low-speed objects which have
significant observational error, the angle can be affected by
significant error.  This means that setting a threshold for angle
tolerance can be difficult; if set too low, then tracklets from
slow-moving objects may not be merged, and if set too high, then
tracklets from fast-moving objects may be mislinked.

This problem would likely be less severe if motion angle and velocity
were replaced by per-axis velocity in RA and declination.  However,
after making this change, it will be necessary to develop
experimentally determine some effective thresholds, which can be a
time-consuming process.

\paragraph{Polar Distortion}
When projecting the locations and motion vectors of objects to a
common time, the current collapseTracklets implementation currently
treats celestial coordinates as though they were planar.  This is
fairly effective close to the ecliptic, but near the poles it will be
problematic.  Fixing this should be relatively trivial, but it has not
yet been done.

\paragraph{Exploring Parallelism and Cacheing of Flags}
As mentioned in \ref{parallelization}, there are several possible
methods for dealing with the per-tracklet ``already merged'' flags.
Currently, the implementation does not explicitly address the issue,
and no attempt has been made to compare various approaches and their
performance.

Currently, there is no explicit cacheing of the ``already merged''
flags, and all threads modify them without regard for locking.  On a
true large-memory, single-board system, this should be fine:
integer-sized reads should be atomic, meaning no partially-written
results could be read, and only in very unusual cases could a stale
value be read. However, in systems with CC-NUMA systems, likely
including vSMP systems such as SDSC's Gordon cluster, it is possible
that stale reads could occur when threads from the same process
execute on multiple boards.


\subsubsection{Dealing with Near-Pole Distortion in LinkTracklets}

The math used in linkTracklets to determine acceleration is almost
certain to break down on near-pole cases.  Further, the chi-squared
probability fitting and filtering has not yet been tested heavily off
the ecliptic, much less in cases of near-pole objects, and will likely
reject true tracks at an increasing rate as it sees objects further
from ecliptic.

To deal with the acceleration pruning near poles, it may be wise to
modify the core piece of code to rotate pairs of nodes to near the
origin at each pruning check; however, this may be too computationally
costly to insert into the ``hot spot'' of the program.  Another
solution would be to abandon celestial coordinates and instead use
full (x,y,z) Cartesian coordinates on a unit sphere.  This will
require new math be used for calculating acceleration and pruning, as
well as for tree construction.  It is unclear how performance will be
affected by the change.

Additional evaluation will be needed before it becomes clear whether
the higher-order fitting code and chi squared probability filter will
be needed whether changes are necessary. 



\subsubsection{Dense and Clustered Noise}

The rate of false tracklet generation increases quadratically on the
density of detections. Issues with image processing can lead to dense
clusters of false detections packed in a small area.  This could lead
to spectacularly painful growth in false tracklets.  

Additional experimentation will be needed to determine how sever this
problem could be.  If it becomes an issue, it may be necessary to
identify regions with unusually high rates of false detections and
simply remove all detections from this region before the data is fed
to MOPS.


