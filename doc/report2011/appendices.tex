\appendix
\section{About the KD-Tree Library}
\label{kdTreeLib}

The findTracklets, collapseTracklets, and linkTracklets algorithms
used in DayMOPS all require the use of KD-Trees for different types of
data.  FindTracklets needs an (RA, Dec) tree of detections,
collapseTracklets needs an (RA, Dec, velocity, angle) tree of
tracklets, and linkTracklets needs a specialized (RA, Dec, RA vel,
Dec vel) tree of tracklets.  Other ``helper'' tools not described in
this document have been constructed using trees over various types of
data (such as collections of Kepler orbits).

To suit the needs of our various algorithms, and to prepare for the
possibility of new algorithms, we chose to create our own KD-Tree
library.  The design is intended to fulfill the following needs:

\begin{itemize}
\item Hold spatial points in arbitrary-dimensional space, with each point mapped to an arbitrary piece of non-spatial data
\item Deal with spatial axes which hold real-number values
\item Deal with spatial axes which hold degree or radian values on a circle
\item Deal with pairs of axes which may describe points on a sphere
\item Allow range searching in circular, spherical or flat Euclidean space
\end{itemize}

The {\tt KDTree} class we have created suits all of these needs.
These features are sufficient for almost all our algorithms.  However,
linkTracklets uses trees in unusual ways and its performance is
closely related to the characteristics of the tree it uses; in order
to allow linkTracklets' special needs, we also created a class called
{\tt TrackletTree} used only for linkTracklets.  Both {\tt KDTree} and
{\tt TrackletTree} derive from a common base class {\tt BaseKDTree}
which implements memory management and other common features.

\subsection{Representing Data Items: PointAndValue Class}

Most DayMOPS algorithms deal with identifying groups of data items
based on their spatial location in some coordinate system.  To
represent a data item and its spatial location, we have created a
template class called {\tt PointAndValue}.  A {\tt PointAndValue}
holds a point in an arbitrary-dimensional coordinate system,
represented as a {\tt std::vector} of {\tt double} values.  The
coordinate system may be arbitrary-dimensional, and may also be
heterogenous; that is, some axes may be Euclidean (having arbitrary
values) but others may be circular, or two axes may describe a sphere.
The ``value'' may be anything - this is a templatized data type.  In
our code, it usually holds an integer, representing the location of a
detection or tracklet in an array.  Using a {\tt PointAndValue}
instance, one can represent any data object (the ``value'' of
arbitrary type) and a location of that data object.

\subsection{Tree Construction and Searching}

{\tt KDTree} instances are constructed from a {\tt std::vector} of
{\tt PointAndValue} instances.  The $k$ value (the dimensionality of
the tree) must be specified at construction-time.  Each ``point''
vector used to build the tree must, of course, have at least $k$
elements, though the software will allow ``points'' with $>k$
elements, in which case only the first $k$ are used.  

When providing the data for points used in tree construction, it is up
to the user to ensure that any value intended to represent a
measurement in degrees must fall along $[0,360)$.  Failure to do so
will lead to an exception later.

If you wish to treat your spatial points as all Euclidean, it is
possible to perform a conventional range search, which searches for
all points in the tree within a given distance of a query point using
{\tt KDTree::rangeSearch}.  However, if your axes use differing units,
or one or more describe points on a circle, range search will not be
suitable, as it does not handle wrap-around in degree units.  This
type of searching is provided by the library, but is not used within
DayMOPS.

In order to deal with data points which may not be in a Euclidean
space, the library provides rectangular (or hyper-rectangular)
searches are supported for searching on mixtures of Euclidean and/or
circular axes.  A query point is provided in each axis, and a query
range around that point, and any point which is within the query range
of the query point in each axis is returned.  This is suitable when
searching, for example, for motion vectors, which have a velocity and
angle of motion; specify a range of velocities, and a range of angles,
and the tree library will find motion vectors which fall within the
given range.  Wraparound will be handled intelligently (that is, the
code will recognize that .1 and 359.9 degrees are separated by .2
degrees, not 359.7.)  To use this type of searching, use the {\tt
KDTree::hyperRectangleSearch} method. Note that it will be necessary
to inform the tree of which axes are Euclidean, and which are
circular.  Currently, the only supported units for circular
coordinates are degrees, and degree measures must fall along $[0,
360)$.  This is the type of searching used in the {\tt
collapseTracklets} implementation, which searches for similar points
in (RA, Dec, velocity and motion angle).  Unfortunately, this type of
searching considers each axis independently, which can be problematic
when dealing with a pair of axes which describe points on a sphere,
where polar distortions may occur.  

To deal with coordinates on a sphere and intelligently account for
both wraparound and polar distortion, the library provides the
function {\tt KDTree::RADecRangeSearch}.  This function performs a
range search on the surface of a sphere; it also allows rectangular
searching of other axes if needed.  The types of the axes should be
specified at searching time as in {\tt KDTree::hyperRectangleSearch},
if any.  This is the search function used by findTracklets, which
represents points as simple RA, Dec coordinates.  It would probably be
wise to change collapseTracklets to use this function as well, though
this has not yet been done.


\subsection{TrackletTree}

The linkTracklets algorithm uses KD-Trees in an unusual fashion, and
is highly sensitive to their construction.  It also requires the
spatial regions held by each tree node be extended to account for
error bars on the tracklets.  For these reasons, linkTracklets does
not use the normal {\tt KDTree} class used by other algorithms, but a
sibling class called {\tt TrackletTree}.  These classes share the
common ancestor {\tt BaseKDTree}, which handles memory management, but
otherwise they share relatively little.  {\tt TrackletTree} calculates
the extents of the nodes differently than {\tt KDTree}, and it
implements no range searching.  

{\tt KDTree} and {\tt TrackletTree} hold references to the head of the
actual tree nodes, implemented by {\tt KDTreeNode} and {\tt
TrackletTreeNode}.  In {\tt KDTree} these child nodes are not visible
to the user; all communication is done with the {\tt KDTree}, which in
turn communicates with the {\tt KDTreeNode}s.  This allows easier
memory management, as an outside tool cannot hold a reference to a
dynamically-allocated child.  However, in linkTracklets these children
must be visible to linkTracklets, as it explicitly traverses the tree.  

{\tt TrackletTreeNode} also has a few methods for studying performance:
{\tt TrackletTreeNode::addVisit} and {\tt TrackletTreeNode::getNumVisits} can
be used to count the number of times a node is examined by
linkTracklets.  Eventually, it may be wise to remove these in order to
conserve memory.


\subsection{BaseKDTree and Memory Management}

{\tt KDTree} and {\tt TrackletTree} represent a whole tree, not the
individual nodes which make up the tree.  {\tt KDTree} then provides
an interface to the querying of the tree, while {\tt TrackletTree}
provides accessors allowing an outside class to see the actual tree
nodes.  

Both classes derive from {\tt BaseKDTree} which handles the mundane
work of a C++ class: construction, copying, and detruction, etc. {\tt
BaseKDTree} handles copying of a tree by incrementing a refcount on
the child nodes; once those nodes hit a refcount of zero, then the
nodes are deleted.  This approach works well, provided no outside
class holds on to references to a tree's nodes after destroying the
tree itself.




\subsection{Needed Improvements}
The KD-Tree nodes at present use quite a bit of memory because they
use {\tt std::vector} where a set of pointers or C-style array would
probably be better-suited.  This is particularly noteworthy in the
case of {\tt TrackletTreeNode}, since trees given to linkTracklets can
be very large.



\section{Helpful Metrics and Software Tools}
\subsection{Studying OpSim}
\subsection{Studying Tracks}

