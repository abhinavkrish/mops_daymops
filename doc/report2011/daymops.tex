\section{DayMOPS}
\label{linking}

\subsection{Approximate Models of Asteroid Motion}
Because of the complexity of the full heliocentric orbital
approximation of asteroid motion, DayMOPS uses simplified models based
on sky-plane motion of asteroid behavior.  

\subsubsection{Real Heliocentric Orbits}
Do we need this?

\subsubsection{Linear and Quadratic Models}
Put Yusra's findings here as well

\subsubsection{Higher-order Sky-Plane Models}
Tim's methods go here - the assumed topocentric distance, topocentric correction, higher-order fits

\subsection{The Linking Pipeline}
Add an illustration of the various stages and a short piece of
introductory text.  Also perhaps useful: show one object's detections
and its various states of linkage. (detections, tracklets, merged tracklets, track(s))

\subsection{Building Tracklets}

% possible illustration: show Dec/time for two images, then tracklets in Dec/time

\textbf{Tracklets} are the building blocks of the sky-plane
\textbf{tracks} used by DayMOPS.  Tracklets are linkages between
DiaSource detections occuring within the same night; during this time
period, solar system object motion is linear or near-linear on the
sky. By creating tracklets, DayMOPS can find sky-plane position and
velocity estimates for sets of detections which may belong to solar
system objects.  This filters out many detections of non-solar system
objects, as they are less likely to generate tracklets.  The use of
tracklets also simplifies the downstream work of track generation,
which attempts to find sets of detections with a good
position/velocity/acceleration fit on the sky-plane; since tracklets
have known position and velocity, the track generation phase needs
only to find those tracklets compatible within some acceleration
factor.

In order to ensure that tracklet-generating images are acquired, it is
necessary to ensure that regions of the sky are visited two or more
times within an accepted time period each night.  Currently, we
require that sky fields be revisited within a fairly short time period
($\leq 90$ minutes is the current rule) in order to constrain the
maximum apparent motion of solar system objects and thus also
constrain the number of tracklets.

The discovery of tracklets can be accomplished efficiently using
KD-Tree structures \citep{bentley_kdtrees} and methods from
\citet{kubica_thesis}.  DayMOPS will build a 2-dimensional (RA, Dec)
KD-Tree for each image, using the tree to hold the detections found in
that image.  Because KD-Trees allow quick range searches of
arbitrary-dimensional spatial data, it is possible to efficiently
perform searches over the detections to find pairs of detections
sufficiently close within time and within limits on apparent
velocity. These pairs of detections are linked to generate tracklets.
% TBD: Is this clear?!

% psuedo code?

Further refinements of tracklets are possible with additional
processing. If an object gets more than one tracklet, it is possible
to use methods similar to the Hough transform to identify and merge
these redundant tracklets into larger tracklets, improving the linear
position/velocity fits of the tracklets and reducing the number of
tracklets passed downstream.

\subsubsection{The findTracklets Software}
Cover the code, its status, what routines do what in the psuedocode, etc.

\subsection{Improving and Filtering Tracklets}
CollapseTracklets issues go here - linking tracklets into longer
tracklets, etc, perhaps some advice on why we do this.

\subsubsection{The collapseTracklets Software}
Cover the code, its status, what routines do what in the psuedocode, etc.

\subsubsection{Tracklet Filtering Software}
What filters we use, and why. Add some information on the code and
status.  Cover RemoveSubsets here.


\subsection{Building Tracks}

Over the course of roughly one month, solar system objects tend to
follow a roughly quadratic path on the sky-plane
\citep{kubica_thesis}.  The track generation phase of DayMOPS will
attempt to find sets of tracklets (which have position and velocity
estimates) which were observed within one month of each other and are
compatible within some acceleration range.  Tracks which are
suitable for generating a reasonable orbital fit are sent to the Orbit
Determination phase.

The methods used for tracklet-to-tracklet linking are described in
\citet{kubica_thesis} and \citet{Kubica:2005:MTA:1081870.1081889}.
The methods described attempt to efficiently find sets of tracklets
which are \textit{compatible} in the sense that they could be joined
to form a track: that is, tracklets which span multiple nights and
have positions and velocities which are consistent with a fixed
acceleration.  

To perform this work efficiently, these methods use four-dimensional
KD-Trees over \textit{tracklet-space}, or (RA position, Dec position,
RA velocity, Dec velocity). One tree is created per image, and holds
each tracklet which has its first detection in that image.  A
multi-tree walk is performed using a clever algorithm, efficiently
discovering all regions of tracklet-space which could contain sets of
tracklets that are compatible, while avoiding visits to tracklet-space
regions which are not compatible and could not generate a track.  This
is performed recursively until leaf nodes of the KD-Trees are reached.

% illustration from Kubica?

When the algorithm encounters a set of leaf nodes in the KD-Trees, it
attempts to build a track using the detections held in the tracklets
at the leaf nodes.  A quadratic fit, or a higher-order fit if
possible, to the detections will be attempted.  Then a quality-of-fit
assessment is used to determine whether the track is considered
sufficiently well-fitted to pass downstream to the Orbit
Determination.  Investigation into ideal higher-order fits and
quality-of-fit metrics is ongoing, but as of this writing a filter on
minimum chi-squared probability appears to be the best option.

\subsubsection{The linkTracklets Software}
Cover the code, its status, what routines do what in the psuedocode,
etc.  Also cover hotspots, sensitive areas, and parameters which have
big impacts on behavior here.


\subsection{Filtering of Tracks}
Some more information on implementation of Tim's fitting and
chi-squared filter and the software. Stats on ground-truth fitting,
possible needs for improvements.


\subsection{Orbit Fitting}
\label{orbitFitting}

Orbits are 6 parameter Keplerian orbits fit to the set of observations
linked during the track building phase. The Keplerian orbit sets
tighter and more complicated constraints on the linkages than the
previous quadratic approximation to this motion, and thus provides a
final filter on linkages which can correspond to true, physical
objects. Using a well-determined orbit, we can predict the location of
an object at arbitrary times.

Orbit fitting can be accomplished using either traditional geometric
methods, where an ellipse or parabola consistent with movement in the
gravitational field of the sun is fit to the set of detections, or
with statistical ranging, where a wide range of potential orbits are
evaluated against the set of detections to search for those with
the lowest residuals. Traditional methods are typically much speedier,
and are available to LSST through the OrbFit software from Milani
\citep{Milani2006}. Statistical ranging methods are more accurate in
exploring the full range of orbital uncertainties for each object,
which can be particularly important for objects observed near 60--90
degrees from the Sun where NEO and MBA exhibit similar apparent
motions, and are available in the OpenOrb software from Granvik
\citep{OpenOrb2009}.

In general, orbit fitting is split into two conceptual pieces - an
``initial orbit determination'' stage, where approximate orbits are
calculated, and a ``differential correction'' stage, where
perturbations on the initial orbit are evaluated to find the best fit
and uncertainty. With six observations on three different nights, most
real moving objects will pass both initial orbit determination and
differential correction with an orbit accurate enough to generate
predicted positions with uncertainties of less than a few arcminutes
for the next few months \citep{basicSolarSystem}.





\subsection{Sky-Plane Motion Limits Imposed by Sky-Plane Linking Methods}

Practical considerations necessitate that we set upper bounds on
tracklet velocity and track acceleration in order to restrict the
number of potential mislinkages. Existing methods attempt to find all
tracklets or tracks within specified velocity and acceleration limits;
as velocity and acceleration limits are raised, the number of
tracklets and tracks can grow quickly.  As a result, the choice of
velocity and acceleration limits is important, as it significantly
impacts the objects found as well as the cost of running the software.

Generally, all types of solar system objects except for the fastest of
near-earth asteroids tend to have reasonably low sky-plane velocity
and acceleration. It is expected that the fastest-moving objects will
leave visible trails in images; these may be used to isolate
detections which could be attributable to fast-movers and restrict the
potential search space for linking these detections.  See section
\ref{neosTrailing} for more information on future plans for
approaching this problem.  

