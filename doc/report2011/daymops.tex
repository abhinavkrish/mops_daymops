\section{DayMOPS}
\label{linking}

\subsection{Approximate Models of Asteroid Motion}
Because of the complexity of the full heliocentric orbital
approximation of asteroid motion, DayMOPS uses simplified models based
on sky-plane motion of asteroid behavior.  


\subsubsection{Real Heliocentric Orbits}
Heliocentric orbits describe an orbit around the Sun.  Generally,
asteroids, planets and other solar system objects follow elliptical
paths, centered on the Sun.  These paths are described (in general
practice as well as the LSST Moving Objects catalog) with a Kepler
orbit, which describes an ellipse using six parameters. 

For purposes of DayMOPS and NightMOPS, we assume that a well-fitted
Kepler orbit should be sufficient to predict the location of an object
well into the future or past.  

\textbf{BD: Illustration of a Kepler orbit - Wikipedia has a good one.}



\subsubsection{Linear and Quadratic Models}
In order to discover and identify new objects, astronomers have
traditionally used sky-plane approximations to predict and model the
behavior of solar system objects for which a true orbit is not yet
known.  As a general rule of thumb, objects are said to move linearly
(with a more or less fixed velocity) in RA and Dec over the course of
a single night and quadratically (having velocity and some
acceleration) in RA and Dec over the course of a month.  These are, of
course, approximations, and linear and quadratic fits will inevitably
contain some error.

Given many detections which may or may not be attributable to one or
more asteroids, these approximations are used to help determine which
detections could plausibly be linked (that is, which detections could
be attributable to the same object).  If several detections over the
course of a single night do not follow a roughly linear path, we trust
that they could not have been attributable to the same object; if
several detections over the course of a month do not follow a roughly
quadratic path, we will trust that they could not be attributable to
the same object.  By ignoring the obviously implausible linkages, we
significantly reduce the number of hypothetical linkages we must
investigate.

Of course, since these are merely approximations, it is almost
inevitable that some correct linkages will be rejected.  In
particular, near-Earth-objects (NEOs) may exhibit sky-plane behaviour
not consistent with these rules of thumb.

\textbf{TBD: Put Yusra's findings here as well}

\subsubsection{Higher-order Sky-Plane Models}
\textbf{TBD: Tim's methods go here - the assumed topocentric distance, topocentric correction, higher-order fits}

\subsection{The Linking Pipeline}
\textbf{TBD: Add an illustration of the various stages and a short piece of
introductory text.  Also perhaps useful: show one object's detections and its various states of linkage. (detections, tracklets, merged tracklets, track(s))}

The LSST DayMOPS system is responsible for finding previously unknown
moving objects in a set of detections.  It takes as input a set of
DiaSource detections with unknown origin.  At output, it should
identify orbits and associated detections for newly-discovered objects
found in the set of input detections.  

The DayMOPS system achieves this through a pipeline approach, building
increasingly sophisticated linkages at each step.  A linkage between a
set of detections represents the hypothesis that these detections
\textit{may} be attributable to the same object.  Unattributed
detections are first linked into nightly linear \textbf{tracklets}.
These tracklets are sets of two or more detections which are roughly
colinear, and have an associated initial sky-plane position and
velocity.  These tracklets are passed forward are passed forward to be
linked into quadratic \textbf{tracks}, each of which contains several
tracklets from multiple nights.  To handle the often large number of
mislinked tracks, a higher-order model is fit the to the detections in
each track and and used to filter them more aggressively.  Those
tracks which survive the filtering are then sent to a full orbit
fitting tool, which should authoritatively reject mislinked sets of
detections and find preliminary orbits associated with
correctly-linked tracks.

Note that tracklets and tracks represent hypothetical linkages, many
of which may be correct.  A single detection may exist in several
tracklets and/or several tracks.  A given tracklet may be found in
multiple tracks.  Some detections may never be linked into any
tracklet; some tracklets will never be linked into any track.  The
DayMOPS system is built using methods and algorithms intended to
efficiently find plausible hypothetical linkages without wasting much
computational time or storage on finding unlikely linkages.

\subsection{Building Tracklets}

\textbf{ possible illustration: show Dec/time for two images, then tracklets in Dec/time}

\textbf{Tracklets} are linkages between DiaSource detections occuring
within the same night. By creating tracklets, DayMOPS can find
sky-plane position and velocity estimates for sets of detections which
may belong to the same solar system objects.  The use of tracklets
also simplifies the downstream work of track generation, which
attempts to find sets of detections with a good
position/velocity/acceleration fit on the sky-plane; since tracklets
have known position and velocity, the track generation phase needs
only to find those tracklets compatible within some acceleration
factor.

Correctly-linked tracklets from a given object are needed to generate
a good track for that object and eventually discover its orbit.
However, if these useful tracklets are too deeply buried among very
large numbers of other tracklets, then the job of tracklet linking
will become extremely slow and expensive.  Generally, these other,
unwanted tracklets are false tracklets (mislinkages between detections
not attributable to the same object), though in special conditions
large numbers of correctly-linked but redundant tracklets can cause
pain as well (this will discussed in \ref{collapseTracklets}).

In order to ensure that tracklet-generating images are acquired, it is
necessary to ensure that fields of the sky are visited two or more
times within an accepted time period each night. To constrain the
number of tracklets, we impose a maximum apparent velocity on the
tracklets, and also require that sky fields be revisited within a
fairly short time period ($\leq 90$ minutes is the current rule).
Raising the maximum velocity threshold enables one to find
faster-moving objects, and raising the maximum allowed revisit time
also enables one to generate tracklets in more fields of the sky;
however, increasing either of these thresholds also increases the
search space and can significantly increase the number of mislinked
tracklets, greatly increasing the cost downstream.




\subsubsection{The findTracklets Software}

The process of initial tracklet creation is accomplished by the
findTracklets software.  Later refinement of tracklets is accomplished
by collapseTracklets and additional filters.

\subsubsubsection{Algorithm} 

The findTracklets software is responsible for finding pairs of
detections which occur within a fixed time threshold, and have
apparent velocity below a given threshold.  Fortunately, this can be
accomplished in a fairly straightforward way through the use of
KD-Trees.  KD-Trees are a data structure which allows for quickly and
efficiently performing range searches on points in space
\citep{bentley_kdtrees}.  A KD-Tree-based method for building
tracklets was first contributed by Jeremy Kubica for his PhD thesis
\citep{kubica_thesis}.  For findTracklets, 2-Dimensional KD-Trees are
used, covering the space of (RA, Dec).  Given a detection and trees
containing detections from later images, we can use range searches to
quickly find nearby detections in those later images and use them for
the creation of tracklets.

\begin{algorithmic}
\REQUIRE $I$ is a set of images, each of which has an associated exposure time and contains a set of detections
\STATE \COMMENT{Create a 2D KD-Tree for each image, holding detections from that image.}
\STATE $T \gets \{\}$
\FOR {$i \in I$}
  \STATE $t \gets$ Make2DTree$(i.detections)$
  \STATE $t.time \gets i.time$
  \STATE $T \gets T \cup \{t\}$
\ENDFOR
\STATE \COMMENT{Use these trees to discover the actual tracklets.}
\STATE $tracklets \gets \{\}$
\FOR {$t_1 \in T$}
  \STATE $later \gets \{t_i \in T : 0 < t_i.time - t_1.time < maxDt\}$
  \FOR{$d \in t_1.detections$}
     \FOR{$t_q \in later$}
 
       \STATE \COMMENT{Use time between images and max velocity to
         calculate the max travel distance}

        \STATE $dt \gets t_q.time - t_1.time$
        \STATE $dd \gets dt * maxV$
        \STATE \COMMENT{Use KD-Tree range search to find detections within max travel distance}
        \STATE $tracklets \gets tracklets \cup t_q.$rangeSearch($d.ra, d.dec, dd$)
     \ENDFOR
   \ENDFOR
\ENDFOR
\RETURN{$tracklets$}
\end{algorithmic}


Cover the code, its status, what routines do what in the psuedocode, etc.

\subsection{Improving and Filtering Tracklets} \label{collapseTracklets}
CollapseTracklets issues go here - linking tracklets into longer
tracklets, etc, perhaps some advice on why we do this.

\subsubsection{The collapseTracklets Software} 
Cover the code, its status, what routines do what in the psuedocode, etc.

\subsubsection{Tracklet Filtering Software}
What filters we use, and why. Add some information on the code and
status.  Cover RemoveSubsets here.


\subsection{Building Tracks}

Over the course of roughly one month, solar system objects tend to
follow a roughly quadratic path on the sky-plane
\citep{kubica_thesis}.  The track generation phase of DayMOPS will
attempt to find sets of tracklets (which have position and velocity
estimates) which were observed within one month of each other and are
compatible within some acceleration range.  Tracks which are
suitable for generating a reasonable orbital fit are sent to the Orbit
Determination phase.

The methods used for tracklet-to-tracklet linking are described in
\citet{kubica_thesis} and \citet{Kubica:2005:MTA:1081870.1081889}.
The methods described attempt to efficiently find sets of tracklets
which are \textit{compatible} in the sense that they could be joined
to form a track: that is, tracklets which span multiple nights and
have positions and velocities which are consistent with a fixed
acceleration.  

To perform this work efficiently, these methods use four-dimensional
KD-Trees over \textit{tracklet-space}, or (RA position, Dec position,
RA velocity, Dec velocity). One tree is created per image, and holds
each tracklet which has its first detection in that image.  A
multi-tree walk is performed using a clever algorithm, efficiently
discovering all regions of tracklet-space which could contain sets of
tracklets that are compatible, while avoiding visits to tracklet-space
regions which are not compatible and could not generate a track.  This
is performed recursively until leaf nodes of the KD-Trees are reached.

% illustration from Kubica?

When the algorithm encounters a set of leaf nodes in the KD-Trees, it
attempts to build a track using the detections held in the tracklets
at the leaf nodes.  A quadratic fit, or a higher-order fit if
possible, to the detections will be attempted.  Then a quality-of-fit
assessment is used to determine whether the track is considered
sufficiently well-fitted to pass downstream to the Orbit
Determination.  Investigation into ideal higher-order fits and
quality-of-fit metrics is ongoing, but as of this writing a filter on
minimum chi-squared probability appears to be the best option.

\subsubsection{The linkTracklets Software}
Cover the code, its status, what routines do what in the psuedocode,
etc.  Also cover hotspots, sensitive areas, and parameters which have
big impacts on behavior here.


\subsection{Filtering of Tracks}
Some more information on implementation of Tim's fitting and
chi-squared filter and the software. Stats on ground-truth fitting,
possible needs for improvements.


\subsection{Orbit Fitting}
\label{orbitFitting}

Orbit fitting can be accomplished using either traditional geometric
methods, where an ellipse or parabola consistent with movement in the
gravitational field of the sun is fit to the set of detections, or
with statistical ranging, where a wide range of potential orbits are
evaluated against the set of detections to search for those with
the lowest residuals. Traditional methods are typically much speedier,
and are available to LSST through the OrbFit software from Milani
\citep{Milani2006}. Statistical ranging methods are more accurate in
exploring the full range of orbital uncertainties for each object,
which can be particularly important for objects observed near 60--90
degrees from the Sun where NEO and MBA exhibit similar apparent
motions, and are available in the OpenOrb software from Granvik
\citep{OpenOrb2009}.

In general, orbit fitting is split into two conceptual pieces - an
``initial orbit determination'' stage, where approximate orbits are
calculated, and a ``differential correction'' stage, where
perturbations on the initial orbit are evaluated to find the best fit
and uncertainty. With six observations on three different nights, most
real moving objects will pass both initial orbit determination and
differential correction with an orbit accurate enough to generate
predicted positions with uncertainties of less than a few arcminutes
for the next few months \citep{basicSolarSystem}.





\subsection{Sky-Plane Motion Limits Imposed by Sky-Plane Linking Methods}

Practical considerations necessitate that we set upper bounds on
tracklet velocity and track acceleration in order to restrict the
number of potential mislinkages. Existing methods attempt to find all
tracklets or tracks within specified velocity and acceleration limits;
as velocity and acceleration limits are raised, the number of
tracklets and tracks can grow quickly.  As a result, the choice of
velocity and acceleration limits is important, as it significantly
impacts the objects found as well as the cost of running the software.

Generally, all types of solar system objects except for the fastest of
near-earth asteroids tend to have reasonably low sky-plane velocity
and acceleration. It is expected that the fastest-moving objects will
leave visible trails in images; these may be used to isolate
detections which could be attributable to fast-movers and restrict the
potential search space for linking these detections.  See section
\ref{neosTrailing} for more information on future plans for
approaching this problem.  

